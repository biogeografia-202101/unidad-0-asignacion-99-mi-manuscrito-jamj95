\documentclass[11pt,]{article}
\usepackage[left=1in,top=1in,right=1in,bottom=1in]{geometry}
\newcommand*{\authorfont}{\fontfamily{phv}\selectfont}
\usepackage[]{mathpazo}


  \usepackage[T1]{fontenc}
  \usepackage[utf8]{inputenc}



\usepackage{abstract}
\renewcommand{\abstractname}{}    % clear the title
\renewcommand{\absnamepos}{empty} % originally center

\renewenvironment{abstract}
 {{%
    \setlength{\leftmargin}{0mm}
    \setlength{\rightmargin}{\leftmargin}%
  }%
  \relax}
 {\endlist}

\makeatletter
\def\@maketitle{%
  \newpage
%  \null
%  \vskip 2em%
%  \begin{center}%
  \let \footnote \thanks
    {\fontsize{18}{20}\selectfont\raggedright  \setlength{\parindent}{0pt} \@title \par}%
}
%\fi
\makeatother




\setcounter{secnumdepth}{3}

\usepackage{longtable,booktabs}

\usepackage{graphicx,grffile}
\makeatletter
\def\maxwidth{\ifdim\Gin@nat@width>\linewidth\linewidth\else\Gin@nat@width\fi}
\def\maxheight{\ifdim\Gin@nat@height>\textheight\textheight\else\Gin@nat@height\fi}
\makeatother
% Scale images if necessary, so that they will not overflow the page
% margins by default, and it is still possible to overwrite the defaults
% using explicit options in \includegraphics[width, height, ...]{}
\setkeys{Gin}{width=\maxwidth,height=\maxheight,keepaspectratio}

\title{``Distribución y abundancia relativa de la familia Rubiaceae en la
parcela permanente Isla Barro Colorado''\\
Subtítulo\\
Subtítulo  }



\author{\Large J. Alberto Meléndez Juan\vspace{0.05in} \newline\normalsize\emph{Universidad Autónoma de Santo Domingo (UASD)}  }


\date{}

\usepackage{titlesec}

\titleformat*{\section}{\normalsize\bfseries}
\titleformat*{\subsection}{\normalsize\itshape}
\titleformat*{\subsubsection}{\normalsize\itshape}
\titleformat*{\paragraph}{\normalsize\itshape}
\titleformat*{\subparagraph}{\normalsize\itshape}

\titlespacing{\section}
{0pt}{36pt}{0pt}
\titlespacing{\subsection}
{0pt}{36pt}{0pt}
\titlespacing{\subsubsection}
{0pt}{36pt}{0pt}





\newtheorem{hypothesis}{Hypothesis}
\usepackage{setspace}

\makeatletter
\@ifpackageloaded{hyperref}{}{%
\ifxetex
  \PassOptionsToPackage{hyphens}{url}\usepackage[setpagesize=false, % page size defined by xetex
              unicode=false, % unicode breaks when used with xetex
              xetex]{hyperref}
\else
  \PassOptionsToPackage{hyphens}{url}\usepackage[unicode=true]{hyperref}
\fi
}

\@ifpackageloaded{color}{
    \PassOptionsToPackage{usenames,dvipsnames}{color}
}{%
    \usepackage[usenames,dvipsnames]{color}
}
\makeatother
\hypersetup{breaklinks=true,
            bookmarks=true,
            pdfauthor={J. Alberto Meléndez Juan (Universidad Autónoma de Santo Domingo (UASD))},
             pdfkeywords = {palabra clave 1, palabra clave 2},  
            pdftitle={``Distribución y abundancia relativa de la familia Rubiaceae en la
parcela permanente Isla Barro Colorado''\\
Subtítulo\\
Subtítulo},
            colorlinks=true,
            citecolor=blue,
            urlcolor=blue,
            linkcolor=magenta,
            pdfborder={0 0 0}}
\urlstyle{same}  % don't use monospace font for urls

% set default figure placement to htbp
\makeatletter
\def\fps@figure{htbp}
\makeatother

\usepackage{amsmath}

\usepackage{pdflscape} \newcommand{\blandscape}{\begin{landscape}} \newcommand{\elandscape}{\end{landscape}}


% add tightlist ----------
\providecommand{\tightlist}{%
\setlength{\itemsep}{0pt}\setlength{\parskip}{0pt}}

\begin{document}
	
% \pagenumbering{arabic}% resets `page` counter to 1 
%
% \maketitle

{% \usefont{T1}{pnc}{m}{n}
\setlength{\parindent}{0pt}
\thispagestyle{plain}
{\fontsize{18}{20}\selectfont\raggedright 
\maketitle  % title \par  

}

{
   \vskip 13.5pt\relax \normalsize\fontsize{11}{12} 
\textbf{\authorfont J. Alberto Meléndez Juan} \hskip 15pt \emph{\small Universidad Autónoma de Santo Domingo (UASD)}   

}

}








\begin{abstract}

    \hbox{\vrule height .2pt width 39.14pc}

    \vskip 8.5pt % \small 

\noindent Resumen del manuscrito


\vskip 8.5pt \noindent \emph{Keywords}: palabra clave 1, palabra clave 2 \par

    \hbox{\vrule height .2pt width 39.14pc}



\end{abstract}


\vskip 6.5pt


\noindent  \section{Introducción}\label{introducciuxf3n}

Las comunidades vegetales de los bosques neotropicales ejemplifican la
diversidad y complejidad ecológica de la región tropical. El estudio
continuo de la riqueza y la abundancia relativa en estas comunidades
permite identificar las especies raras, las cuales son más vulnerables a
los cambios en su hábitat y por lo tanto propensas a extinguirse
localmente (Volkov, Banavar, Hubbell, \& Maritan, 2003). Conocer estos
aspectos de las comunidades ecológicas y como se encuentran distribuidas
en el espacio las especies que las componen, ofrece la oportunidad de
comprender como evolucionan en el tiempo y los factores que inciden en
su conservación (Moreno, 2001).

La familia Rubiaceae es un importante grupo de plantas vasculares de
distribución cosmopolita con una marcada diversidad en regiones
tropicales y subtropicales (Davis et al., 2009). Muchas de las especies
que componen esta familia se encuentran adaptadas a la vida en la
penúmbra, y prosperan bajo la sombra del dosel selvatico neotropical,
región en la cual son especialmente diversas. En estas selvas
tropicales, la organización de las comunidades de plantas del sotobosque
depende en gran medida de las interacciónes existentes entre las
distintas especies que lo habitan (Torres-Leite et al., 2019). Es
preciso señalar que estudios anteriores realizados en el bosque tropical
panameño sobre el grado de reemplazo entre especies de distintas
comunidades o diversidad beta, sugieren una tendencia a la disimilaridad
entre comunidades en cuanto a su composición, esta aumentando con la
distancia en la cual se encuentran separadas en el espacio (Condit et
al., 2002). Sin embargo, estos trabajos no restan importancia a la
variabilidad del hábitat y se estima su importancia en el estudio de la
composición de estos ecosistemas.

.Si bien es cierto que la distribución de la abundancia de especies
depende de características que definen una comunidad en particular,
existiendo una proporción variable de especies dominantes, con una
abundancia alta en comparación con las especies raras y menos
abundantes, las medidas para la distribución de la abundancia relativa
se encuentran sujetas a diversas variables las cuales aún no se conocen
del todo ni en qué grado inciden en la estructura de la comunidad (Néda,
Horvat, Toháti, Derzsi, \& Balogh, 2008).

El presente estudio intenta la relación entre abundancia relativa de
especies de la familia rubiaceae y su distribución en una porción de
bosque húmedo tropical en la parcela permanente Barro Colorado Island
(BCI), ubicada en la provincia Colón, Panamá. Los parámetros de riqueza
y abundancia relativa obtenidos mediante análisis de datos de los censos
realizados en Barro Colorado contribuyen a medir el aporte de la familia
rubiaceae a la diversidad de su comunidad. En ese sentido, este trabajo
aprovecha la información disponible sobre las características del
hábitat en el cual crecen estas poblaciones de plantas (Hubbell, Condit,
\& Foster, 2021) para conocer posibles patrones en la distribución de
las especies y como varía la diversidad alpha con respecto a propiedades
del terreno y otras condiciones ambientales medibles.

\section{Metodología}\label{metodologuxeda}

La parcela permanente BCI es una estación de censo permanente
administrada por el Instituto Smithsoniano de Investigaciones Tropicales
ubicada en el centro de la isla Barro Colorado en la cuenca del canal de
Panamá, con las coordenadas 09\(^\circ\)~09'N, 079\(^\circ\)~50'O. Es un
polígono de 50 hectáreas cuadradas en el cual se han contabilizado todos
los arboles con más de 10 mm de diámetro a la altura del pecho cada
cinco años desde 1985 (Hubbell \& Foster, 1983, Hubbell et al. (1990),
Condit, Chisholm, \& Hubbell (2012), Condit, Pérez, Lao, Aguilar, \&
Hubbell (2017)); en este estudio se utilizaron las datos del censo
realizado en 2015.

Los datos referentes a estos censos fueron manejados en R (R Core Team
(2020)) partiendo de su disposición en dos matrices de comunidad y
ambiental de cada uno de los 50 cuadrantes de una hectárea que componen
BCI (Martínez Batlle, 2020). Estas matrices contienen datos de las
variables ambientales como composición química del suelo, tipo de
hábitat, geomorfología y edad geomorfológica. Así como datos
demográficos y de ubicación espacial de todos los individuos censados.
Se adaptaron \emph{scripts} reproducibles recuperados de Martínez Batlle
(2020), utilizando la colección de paquetes multifuncionales
\emph{Tidyverse} (Wickham, 2017), paquetes gráficos y de procesamiento
de datos espaciales para la representación de mapas y figuras como
\texttt{mapview} (Appelhans, Detsch, Reudenbach, \& Woellauer, 2019) y
\emph{simplefeatures} (Pebesma, 2018); y herramientas de análisis
estadístico como \texttt{vegan} (Oksanen et al., 2019),
\texttt{indicspecies} (De Caceres \& Legendre, 2009), entre otros (ver
\ref{información suplementaria}). \dots

Para conocer las características distintivas de los datos conservados en
las matrices de comunidad y ambiental, se realizó un análisis
exploratorio de los mismos que incluyó un resumen estadístico (media
aritmética y mediana) de la riqueza númerica de especies, la abundancia
y de las variables ambientales tomadas en BCI. También se realizaron
análisis gráficos con el apoyo de tablas, mapas de los cuadrantes y
paneles para el análisis de correlación lineal entre variables de ambas
matrices, con el fin de obtener una perspectiva general y ayudar a
determinar los procedimientos posteriores que se detallan acontinuación
\dots

pruebas de medición de asociación, para lo que se calculó la distancia
de Hellinger entre los cuadrantes considerados como objetos. Para esto,
fué requerida la transformada de la matríz de comunidad por el método de
Hellinger, el cual consiste en la radicación al cuadrado de la
abundancia relativa \[ y_{ij}\] como muestra la fórmula
\ref{eq:hell_transf}, donde \emph{j} refiere a cada especie o columna en
la \textbf{matríz}, \emph{i} es la fila o cuadrante e \emph{i+}
representa la suma de filas de la matríz del i-ésimo cuadrante (Legendre
\& Gallagher, 2001). Además, la distancia euclidea entre cuadrantes en
cuanto a la presencia de especies fué evaluada aplicando el índice de
disimilaridad de Sorensen de la matríz normalizada, con valores de
abundancia convertidos en valores binarios.\\
\dots

\begin{equation} \label{eq:hell_transf}
y' = \sqrt{\frac{y_{ij}}{y_{i+}}}
\end{equation}

\dots
Para conocer la relación de la ocurrencia entre las especies y su
distribución en BCI, se utilizó el coeficiente de Spearman para medir el
grado de correlación entre las variables riqueza númerica de especies y
la abundancia con las variables ambientales geomorfológicas y la
composición química del suelo.

\section{Resultados}\label{resultados}

\dots
La familia Rubiaceae en Barro Colorado se encuentra representada por 31
especies y 20 generos. El género \emph{Psychotria} presenta la mayor
cantidad de especies con 8?. La tabla \ref{tab:abun_sp} indica las
abundancias de las especies de toda la comunidad que en total suman
41,838 individuos, con una abundancia media de 65 individuos y mediana
ubicada en los 1,350 individuos. El mapa de cuadros de la figura
\ref{fig:mapa_cuadros_riq} muestra la riqueza numérica de especies por
cuadrante. Los valores de abundancia muestran un aparente patrón en
algunos lugares de BCI, al presentar además el valor máximo en riqueza
de la familia {[}ver figura \ref{fig:mapa_cuadros_abun_rubic}{]}.

Los valores para el coeficiente de Spearman de correlación entre rangos
no mostraron evidencia de que exista relación entre la riqueza numérica
de especies y la abundancia con las variables geomorfológicas notadas en
la matríz de variables ambientales. Sin embargo, el mismo análisis
sugiere una posible relación entre la abundancia y la compososición del
suelo, mostrando relación positiva con valores altos de Aluminio y
Fósforo, así como negativa, para valores altos de pH y concentraciones
de otros elementos (B, Ca, Cu, Fe, K, Mg, Mn, Zn y Nitrógeno
mineralizado). \#ad

\begin{longtable}[]{@{}lr@{}}
\caption{\label{tab:abun_sp}Abundancia por especie.}\tabularnewline
\toprule
Latin & n\tabularnewline
\midrule
\endfirsthead
\toprule
Latin & n\tabularnewline
\midrule
\endhead
Faramea occidentalis & 24989\tabularnewline
Alseis blackiana & 7928\tabularnewline
Psychotria horizontalis & 2453\tabularnewline
Coussarea curvigemmia & 2010\tabularnewline
Palicourea guianensis & 1118\tabularnewline
Randia armata & 937\tabularnewline
Psychotria marginata & 761\tabularnewline
Alibertia edulis & 417\tabularnewline
Pentagonia macrophylla & 306\tabularnewline
Guettarda foliacea & 252\tabularnewline
Hamelia axillaris & 128\tabularnewline
Macrocnemum roseum & 87\tabularnewline
Posoqueria latifolia & 73\tabularnewline
Psychotria limonensis & 70\tabularnewline
Genipa americana & 67\tabularnewline
Psychotria graciliflora & 65\tabularnewline
Psychotria grandis & 57\tabularnewline
Psychotria deflexa & 38\tabularnewline
Amaioua corymbosa & 19\tabularnewline
Psychotria chagrensis & 16\tabularnewline
Psychotria acuminata & 14\tabularnewline
Tocoyena pittieri & 8\tabularnewline
Psychotria racemosa & 7\tabularnewline
Psychotria cyanococca & 4\tabularnewline
Chimarrhis parviflora & 3\tabularnewline
Coutarea hexandra & 3\tabularnewline
Psychotria brachiata & 3\tabularnewline
Appunia seibertii & 2\tabularnewline
Borojoa panamensis & 1\tabularnewline
Psychotria hoffmannseggiana & 1\tabularnewline
Rosenbergiodendron formosum & 1\tabularnewline
\bottomrule
\end{longtable}

\begin{figure}
\centering
\includegraphics{manuscrito_files/figure-latex/unnamed-chunk-3-1.pdf}
\caption{\label{fig:abun_sp_q}Número de individuos de cada especie por
cuadrante.}
\end{figure}

\begin{figure}
\centering
\includegraphics{mapa_cuadros_abun_rubic.png}
\caption{Abundancia de rubiaceas en BCI
\label{fig:mapa_cuadros_abun_rubic}}
\end{figure}

\begin{figure}
\centering
\includegraphics{mapa_cuadros_riq_rubic.png}
\caption{Distribución de la riqueza de rubiaceas en BCI
\label{fig:mapa_cuadros_riq}}
\end{figure}

\begin{figure}
\centering
\includegraphics{mapa_cuadros_ph.png}
\caption{pH del suelo en los distintos cuadros de 1ha
\label{fig:mapa_cuadros_ph}}
\end{figure}

\section{Discusión}\label{discusiuxf3n}

\section{Agradecimientos}\label{agradecimientos}

\section{Información de soporte}\label{informaciuxf3n-de-soporte}

\ldots

\section{\texorpdfstring{\emph{Script}
reproducible}{Script reproducible}}\label{script-reproducible}

\ldots

\section*{Referencias}\label{referencias}
\addcontentsline{toc}{section}{Referencias}

\hypertarget{refs}{}
\hypertarget{ref-cita_mapview}{}
Appelhans, T., Detsch, F., Reudenbach, C., \& Woellauer, S. (2019).
\emph{Mapview: Interactive viewing of spatial data in r}. Retrieved from
\url{https://CRAN.R-project.org/package=mapview}

\hypertarget{ref-condit_et_al_2012}{}
Condit, R., Chisholm, R. A., \& Hubbell, S. P. (2012). Thirty years of
forest census at Barro Colorado and the importance of immigration in
maintaining diversity. \emph{PLOS ONE}, \emph{7}(11), 1--6.
\url{https://doi.org/10.1371/journal.pone.0049826}

\hypertarget{ref-condit_et_al_2017}{}
Condit, R., Pérez, R., Lao, S., Aguilar, S., \& Hubbell, S. P. (2017).
Demographic trends and climate over 35 years in the Barro Colorado 50 ha
plot. \emph{Forest Ecosystems}, \emph{4}(1), 17.
\url{https://doi.org/10.1186/s40663-017-0103-1}

\hypertarget{ref-article_condit}{}
Condit, R., Pitman, N., Leigh, E., Chave, J., Terborgh, J., Foster, R.,
\ldots{} Hubbell, S. (2002). Beta-diversity in tropical forest trees.
\emph{Science (New York, N.Y.)}, \emph{295}, 666--669.
\url{https://doi.org/10.1126/science.1066854}

\hypertarget{ref-davis2009global}{}
Davis, A. P., Govaerts, R., Bridson, D. M., Ruhsam, M., Moat, J., \&
Brummitt, N. A. (2009). A global assessment of distribution, diversity,
endemism, and taxonomic effort in the rubiaceae1. \emph{Annals of the
Missouri Botanical Garden}, \emph{96}(1), 68--78.

\hypertarget{ref-cita_indicspecies}{}
De Caceres, M., \& Legendre, P. (2009). Associations between species and
groups of sites: Indices and statistical inference. In \emph{Ecology}.
Retrieved from \url{http://sites.google.com/site/miqueldecaceres/}

\hypertarget{ref-hubell_foster_1983}{}
Hubbell, S. P., \& Foster, R. B. (1983). Diversity of canopy trees in a
neotropical forest and implications for conservation. In T. Whitmore, A.
Chadwick, \& A. Sutton (Eds.), \emph{Tropical rain forest: Ecology and
management} (pp. 25--41). Oxford: The British Ecological Society.

\hypertarget{ref-hubell_et_all_1990}{}
Hubbell, S. P., Condit, R., Foster, R. B., Grubb, P. J., Thomas, C. D.,
Hassell, M. P., \& May, R. M. (1990). Presence and absence of density
dependence in a neotropical tree community. \emph{Philosophical
Transactions of the Royal Society of London. Series B: Biological
Sciences}, \emph{330}(1257), 269--281.
\url{https://doi.org/10.1098/rstb.1990.0198}

\hypertarget{ref-web_bci}{}
Hubbell, S., Condit, R., \& Foster, R. (2021). Forest Census Plot on
Barro Colorado Island. Retrieved March 23, 2021, from
\url{http://ctfs.si.edu/webatlas/datasets/bci/}

\hypertarget{ref-legendre_galllagher_2001}{}
Legendre, P., \& Gallagher, E. (2001). Ecologically meaningful
transformations for ordination of species data. \emph{Oecologia},
\emph{129}, 271--280. \url{https://doi.org/10.1007/s004420100716}

\hypertarget{ref-jose_ramon_martinez_batlle_2020_4402362}{}
Martínez Batlle, J. R. (2020).
biogeografia-master/scripts-de-analisis-BCI: Long coding sessions
(Version v0.0.0.9000). \url{https://doi.org/10.5281/zenodo.4402362}

\hypertarget{ref-moreno2001manual}{}
Moreno, C. E. (2001). \emph{Manual de métodos para medir la
biodiversidad}. Universidad Veracruzana.

\hypertarget{ref-2008arXiv0803.3704N}{}
Néda, Z., Horvat, S., Toháti, H. M., Derzsi, A., \& Balogh, A. (2008). A
spatially explicit model for tropical tree diversity patterns.
\emph{arXiv E-Prints}, arXiv:0803.3704.

\hypertarget{ref-cita_vegan}{}
Oksanen, J., Blanchet, F. G., Friendly, M., Kindt, R., Legendre, P.,
McGlinn, D., \ldots{} Wagner, H. (2019). \emph{Vegan: Community ecology
package}. Retrieved from \url{https://CRAN.R-project.org/package=vegan}

\hypertarget{ref-cita_sf}{}
Pebesma, E. (2018). Simple Features for R: Standardized Support for
Spatial Vector Data. \emph{The R Journal}, \emph{10}(1), 439--446.
\url{https://doi.org/10.32614/RJ-2018-009}

\hypertarget{ref-cita_r}{}
R Core Team. (2020). \emph{R: A language and environment for statistical
computing}. Retrieved from \url{https://www.R-project.org/}

\hypertarget{ref-TORRESLEITE2019151487}{}
Torres-Leite, F., Cavatte, P. C., Garbin, M. L., Hollunder, R. K.,
Ferreira-Santos, K., Capetine, T. B., \ldots{} Carrijo, T. T. (2019).
Surviving in the shadows: Light responses of co-occurring rubiaceae
species within a tropical forest understory. \emph{Flora}, \emph{261},
151487.
\url{https://doi.org/https://doi.org/10.1016/j.flora.2019.151487}

\hypertarget{ref-Volkov_2003}{}
Volkov, I., Banavar, J. R., Hubbell, S. P., \& Maritan, A. (2003).
Neutral theory and relative species abundance in ecology. \emph{Nature},
\emph{424}(6952), 1035--1037. \url{https://doi.org/10.1038/nature01883}

\hypertarget{ref-cita_tidyverse}{}
Wickham, H. (2017). \emph{Tidyverse: Easily install and load the
'tidyverse'}. Retrieved from
\url{https://CRAN.R-project.org/package=tidyverse}




\newpage
\singlespacing 
\end{document}
