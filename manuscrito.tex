\documentclass[11pt,]{article}
\usepackage[left=1in,top=1in,right=1in,bottom=1in]{geometry}
\newcommand*{\authorfont}{\fontfamily{phv}\selectfont}
\usepackage[]{mathpazo}


  \usepackage[T1]{fontenc}
  \usepackage[utf8]{inputenc}



\usepackage{abstract}
\renewcommand{\abstractname}{}    % clear the title
\renewcommand{\absnamepos}{empty} % originally center

\renewenvironment{abstract}
 {{%
    \setlength{\leftmargin}{0mm}
    \setlength{\rightmargin}{\leftmargin}%
  }%
  \relax}
 {\endlist}

\makeatletter
\def\@maketitle{%
  \newpage
%  \null
%  \vskip 2em%
%  \begin{center}%
  \let \footnote \thanks
    {\fontsize{18}{20}\selectfont\raggedright  \setlength{\parindent}{0pt} \@title \par}%
}
%\fi
\makeatother




\setcounter{secnumdepth}{3}

\usepackage{longtable,booktabs}

\usepackage{graphicx,grffile}
\makeatletter
\def\maxwidth{\ifdim\Gin@nat@width>\linewidth\linewidth\else\Gin@nat@width\fi}
\def\maxheight{\ifdim\Gin@nat@height>\textheight\textheight\else\Gin@nat@height\fi}
\makeatother
% Scale images if necessary, so that they will not overflow the page
% margins by default, and it is still possible to overwrite the defaults
% using explicit options in \includegraphics[width, height, ...]{}
\setkeys{Gin}{width=\maxwidth,height=\maxheight,keepaspectratio}

\title{Distribución y abundancia relativa de la familia Rubiaceae en la parcela
permanente Isla Barro Colorado\\
Subtítulo\\
Subtítulo  }



\author{\Large J. Alberto Meléndez Juan\vspace{0.05in} \newline\normalsize\emph{Universidad Autónoma de Santo Domingo (UASD)}  }


\date{}

\usepackage{titlesec}

\titleformat*{\section}{\normalsize\bfseries}
\titleformat*{\subsection}{\normalsize\itshape}
\titleformat*{\subsubsection}{\normalsize\itshape}
\titleformat*{\paragraph}{\normalsize\itshape}
\titleformat*{\subparagraph}{\normalsize\itshape}

\titlespacing{\section}
{0pt}{36pt}{0pt}
\titlespacing{\subsection}
{0pt}{36pt}{0pt}
\titlespacing{\subsubsection}
{0pt}{36pt}{0pt}





\newtheorem{hypothesis}{Hypothesis}
\usepackage{setspace}

\makeatletter
\@ifpackageloaded{hyperref}{}{%
\ifxetex
  \PassOptionsToPackage{hyphens}{url}\usepackage[setpagesize=false, % page size defined by xetex
              unicode=false, % unicode breaks when used with xetex
              xetex]{hyperref}
\else
  \PassOptionsToPackage{hyphens}{url}\usepackage[unicode=true]{hyperref}
\fi
}

\@ifpackageloaded{color}{
    \PassOptionsToPackage{usenames,dvipsnames}{color}
}{%
    \usepackage[usenames,dvipsnames]{color}
}
\makeatother
\hypersetup{breaklinks=true,
            bookmarks=true,
            pdfauthor={J. Alberto Meléndez Juan (Universidad Autónoma de Santo Domingo (UASD))},
             pdfkeywords = {palabra clave 1, palabra clave 2},  
            pdftitle={Distribución y abundancia relativa de la familia Rubiaceae en la parcela
permanente Isla Barro Colorado\\
Subtítulo\\
Subtítulo},
            colorlinks=true,
            citecolor=blue,
            urlcolor=blue,
            linkcolor=magenta,
            pdfborder={0 0 0}}
\urlstyle{same}  % don't use monospace font for urls

% set default figure placement to htbp
\makeatletter
\def\fps@figure{htbp}
\makeatother

\usepackage{pdflscape} \newcommand{\blandscape}{\begin{landscape}}
\newcommand{\elandscape}{\end{landscape}}


% add tightlist ----------
\providecommand{\tightlist}{%
\setlength{\itemsep}{0pt}\setlength{\parskip}{0pt}}

\begin{document}
	
% \pagenumbering{arabic}% resets `page` counter to 1 
%
% \maketitle

{% \usefont{T1}{pnc}{m}{n}
\setlength{\parindent}{0pt}
\thispagestyle{plain}
{\fontsize{18}{20}\selectfont\raggedright 
\maketitle  % title \par  

}

{
   \vskip 13.5pt\relax \normalsize\fontsize{11}{12} 
\textbf{\authorfont J. Alberto Meléndez Juan} \hskip 15pt \emph{\small Universidad Autónoma de Santo Domingo (UASD)}   

}

}








\begin{abstract}

    \hbox{\vrule height .2pt width 39.14pc}

    \vskip 8.5pt % \small 

\noindent Resumen del manuscrito


\vskip 8.5pt \noindent \emph{Keywords}: palabra clave 1, palabra clave 2 \par

    \hbox{\vrule height .2pt width 39.14pc}



\end{abstract}


\vskip 6.5pt


\noindent  \section{Introducción}\label{introducciuxf3n}

Las comunidades vegetales de los bosques neotropicales ejemplifican la
diversidad y complejidad ecológica de la región tropical. El estudio
continuo de la riqueza y la abundancia relativa en estas comunidades
permite identificar las especies raras, las cuales son más vulnerables a
los cambios en su hábitat y propensas a extinguirse localmente (Volkov,
Banavar, Hubbell, \& Maritan, 2003). Conocer estos aspectos de las
comunidades ecológicas y como se encuentran distribuidas en el espacio
las especies que las componen, ofrece la oportunidad de comprender como
evolucionan en el tiempo y los factores que inciden en su conservación
(Moreno, 2001).

La familia Rubiaceae es un importante grupo de plantas vasculares de
distribución cosmopolita con una marcada diversidad en regiones
tropicales y subtropicales (Davis et al., 2009). Las rubieaceas son
especialmente diversas en el neotrópico y resulta un grupo idóneo para
estudios ecológicos con énfasis en la estructura de las comunidades de
la región. El presente estudio intenta averiguar la relación entre
abundancia relativa de especies de la familia rubiaceae y su
distribución en una porción de bosque húmedo tropical en la parcela
permanente Barro Colorado Island, Panamá-----{[}{]}. Los parámetros de
riqueza y abundancia relativa obtenidos mediante análisis de datos de
los censos realizados en el área de estudio indicarán el aporte de la
familia rubiaceae a la diversidad de su comunidad.

Muchas especies dentro de la familia rubiaceae se encuentran adaptadas a
rangos elevados de ácides y otras condiciones específicas de los
componentes del suelo, como diversos grados de concentración de
distintos metales(Jansen, Robbrecht, Beeckman, \& Smets, 2000). En ese
sentido, este trabajo aprovecha la información disponible sobre las
características del hábitat en el cual crecen las poblaciones de plantas
para conocer posibles patrones en la distribución de las especies objeto
de este estudio y como varía la diversidad alpha con respecto a estas
propiedades del terreno y otras condiciones ambientales medibles.

Además de las condiciones ambientales, estudios anteriores sugieren que
interacciones interespecíficas entre rubiaceas del sotobosque amazonico
(Torres-Leite et al. (2019)) podrían determinar como se organizan estas
especies de plantas en la comunidad vegetal y conocer la relación entre
la ocurrencia de las distintas especies de rubiaceas en nuestra área de
estudio forma parte de los objetivos de este trabajo.

Estudios anteriores de la diversidad beta de la zona, especificamente
del bosque tropical panameño, sugieren una tendencia a la disimilaridad
en la composición de las comunidades entre sí que aumenta con la
distancia en la cual las comunidades se encuentran separadas en el
espacio. (comunidades comparadas lejanas unas de otras son más
diferentes en cuanto a su composición floristica).

Como se estructuran las comunidades ecológicas refiere al número de
individuos de cada taxa que las componen(Ricotta, 2004), además de su
distribución en el espacio. Este número se encuentra sujeto a diversas
variables las cuales aún no se conocen del todo ni en qué grado inciden
en la estructura de la comunidad (Neda, Horvat, Tohati, Derzsi, \&
Balogh, 2008).

\section{Metodología}\label{metodologuxeda}

\ldots

\section{Resultados}\label{resultados}

Como indica la tabla \ref{tab:abun_sp}

\begin{longtable}[]{@{}lr@{}}
\caption{\label{tab:abun_sp}Abundancia por especie.}\tabularnewline
\toprule
Latin & n\tabularnewline
\midrule
\endfirsthead
\toprule
Latin & n\tabularnewline
\midrule
\endhead
Faramea occidentalis & 24989\tabularnewline
Alseis blackiana & 7928\tabularnewline
Psychotria horizontalis & 2453\tabularnewline
Coussarea curvigemmia & 2010\tabularnewline
Palicourea guianensis & 1118\tabularnewline
Randia armata & 937\tabularnewline
Psychotria marginata & 761\tabularnewline
Alibertia edulis & 417\tabularnewline
Pentagonia macrophylla & 306\tabularnewline
Guettarda foliacea & 252\tabularnewline
Hamelia axillaris & 128\tabularnewline
Macrocnemum roseum & 87\tabularnewline
Posoqueria latifolia & 73\tabularnewline
Psychotria limonensis & 70\tabularnewline
Genipa americana & 67\tabularnewline
Psychotria graciliflora & 65\tabularnewline
Psychotria grandis & 57\tabularnewline
Psychotria deflexa & 38\tabularnewline
Amaioua corymbosa & 19\tabularnewline
Psychotria chagrensis & 16\tabularnewline
Psychotria acuminata & 14\tabularnewline
Tocoyena pittieri & 8\tabularnewline
Psychotria racemosa & 7\tabularnewline
Psychotria cyanococca & 4\tabularnewline
Chimarrhis parviflora & 3\tabularnewline
Coutarea hexandra & 3\tabularnewline
Psychotria brachiata & 3\tabularnewline
Appunia seibertii & 2\tabularnewline
Borojoa panamensis & 1\tabularnewline
Psychotria hoffmannseggiana & 1\tabularnewline
Rosenbergiodendron formosum & 1\tabularnewline
\bottomrule
\end{longtable}

\begin{figure}
\centering
\includegraphics{manuscrito_files/figure-latex/unnamed-chunk-3-1.pdf}
\caption{\label{fig:abun_sp_q}Número de individuos de cada especie por
hectárea.}
\end{figure}

\begin{longtable}[]{@{}lllll@{}}
\toprule
c1 & c2 & c3 & c4 & c5\tabularnewline
\midrule
\endhead
adasdas & adasdadsadaadadadasdasdsau dadasdadadasdasdsaad & adasd & aasd
& asdsd\tabularnewline
adasddd & asadasdasd & adsadas & adad & asdasddad\tabularnewline
\bottomrule
\end{longtable}

\section{Discusión}\label{discusiuxf3n}

\section{Agradecimientos}\label{agradecimientos}

\section{Información de soporte}\label{informaciuxf3n-de-soporte}

\ldots

\section{\texorpdfstring{\emph{Script}
reproducible}{Script reproducible}}\label{script-reproducible}

\ldots

\section*{Referencias}\label{referencias}
\addcontentsline{toc}{section}{Referencias}

\hypertarget{refs}{}
\hypertarget{ref-davis2009global}{}
Davis, A. P., Govaerts, R., Bridson, D. M., Ruhsam, M., Moat, J., \&
Brummitt, N. A. (2009). A global assessment of distribution, diversity,
endemism, and taxonomic effort in the rubiaceae1. \emph{Annals of the
Missouri Botanical Garden}, \emph{96}(1), 68--78.

\hypertarget{ref-article}{}
Jansen, S., Robbrecht, E., Beeckman, H., \& Smets, E. (2000). Aluminium
accumulation in rubiaceae: An additional character for the delimitation
of the subfamily rubioideae? \emph{IAWA Journal}, \emph{21}.
\url{https://doi.org/10.1163/22941932-90000245}

\hypertarget{ref-moreno2001manual}{}
Moreno, C. E. (2001). \emph{Manual de métodos para medir la
biodiversidad}. Universidad Veracruzana.

\hypertarget{ref-2008arXiv0803.3704N}{}
Neda, Z., Horvat, S., Tohati, H. M., Derzsi, A., \& Balogh, A. (2008). A
spatially explicit model for tropical tree diversity patterns.
\emph{arXiv E-Prints}, arXiv:0803.3704.

\hypertarget{ref-https:ux2fux2fdoi.orgux2f10.1111ux2fj.1366-9516.2004.00069.x}{}
Ricotta, C. (2004). A parametric diversity measure combining the
relative abundances and taxonomic distinctiveness of species.
\emph{Diversity and Distributions}, \emph{10}(2), 143--146.
\url{https://doi.org/https://doi.org/10.1111/j.1366-9516.2004.00069.x}

\hypertarget{ref-TORRESLEITE2019151487}{}
Torres-Leite, F., Cavatte, P. C., Garbin, M. L., Hollunder, R. K.,
Ferreira-Santos, K., Capetine, T. B., \ldots{} Carrijo, T. T. (2019).
Surviving in the shadows: Light responses of co-occurring rubiaceae
species within a tropical forest understory. \emph{Flora}, \emph{261},
151487.
\url{https://doi.org/https://doi.org/10.1016/j.flora.2019.151487}

\hypertarget{ref-Volkov_2003}{}
Volkov, I., Banavar, J. R., Hubbell, S. P., \& Maritan, A. (2003).
Neutral theory and relative species abundance in ecology. \emph{Nature},
\emph{424}(6952), 1035--1037. \url{https://doi.org/10.1038/nature01883}




\newpage
\singlespacing 
\end{document}
